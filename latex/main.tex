
\RequirePackage{amsmath}
\documentclass[runningheads]{llncs}
%
\usepackage{graphicx}
\usepackage[utf8]{inputenc} \usepackage[T1]{fontenc} 
\usepackage{url}
\usepackage{hyperref}
\usepackage{color}
\usepackage{soul}
%\usepackage[linesnumbered,ruled]{algorithm2e}
\usepackage{a4wide}
\usepackage{amsmath}
% If you use the hyperref package, please uncomment the following line
% to display URLs in blue roman font according to Springer's eBook style:
 \renewcommand\UrlFont{\color{blue}\rmfamily}
%\newcommand{\red}[1]{\textcolor{red}{#1}}
\newcommand{\red}[1]{#1}

\begin{document}
%
\title{Optimal Solutions to Common Interview Problems}
\subtitle{A Summary of Chapter II.Problems of the EPI book, as well as stuff I found interesting}

\author{Sarah Lutteropp}

\authorrunning{S. Lutteropp}

\maketitle

\begin{abstract}

In this document, I summarize the main optimal solution approaches for common coding interview problems. This document is meant for personal use. (And for making my handwritten notes easier to read \ldots)

\end{abstract}

\section{Interview Toolbox}

\subsection{Primitive Types / Bit Tricks}

\paragraph{Remove lowest set bit in $x$}
$$x = x \& (x-1)$$

\paragraph{Isolate lowest set bit in $x$}
$$x \& ~(x-1)$$

\paragraph{Extract i-th bit of $x$}
$$x \& (1 << i)$$

\paragraph{Swap bits}
Only needed if the bits are not the same. If needed, it's like flipping bits:
$$x \wedge = ((1 << i) | (1 << j))$$

\paragraph{Reverse bits}
Use precomputed lookup table for 16-bit words.
If $x = y_1 y_2 y_3 y_4$, then $\text{rev}(x) = \text{rev}(y_4) \text{rev}(y_3) \text{rev}(y_2) \text{rev}(y_1) $.
Useful for this: $0xFFFF = 65535 = 2^{16}-1 = 1111.1111.1111.1111$ (16 ones)

\paragraph{Closest integer with same number of ones}
If we flip bits at index $i$ and $j$, the absolute difference is $|2^i - 2^j|$.
Thus, we need to swap the two rightmost consecutive bits that differ.

\paragraph{$x*y$ without arithmetic operations}
Add using bitwise operations. Multiply using shift-and-add (grade-school) algorithm.

\paragraph{$x/y$ using only +, -, and bitshifts}
Idea: Find largest $k$ such that $2^k * y \leq x$, substract $2^k$ fom $x$, and add $2^k$ to the quotient.

\paragraph{Compute $x^y$}
If $y$ is negative, replace $x$ by $1/x$ and $y$ by $-y$.
Idea: $(x^\frac{y}{2})^2$ or $x * (x^\frac{y}{2})^2$

\paragraph{Properties of XOR}
\begin{itemize}
\item If two numbers where both numbers have odd number of set bits in it then its XOR will have even number of set bits in it.
\item If two numbers where both numbers have even number of set bits in it then its XOR will have even number of set bits in it.
\item If two numbers where one number has even number of set bits and another has odd number of set bits then its XOR will have odd number of set bits in it.
\end{itemize}

\paragraph{Check if decimal is palindrome}
A negative number can't be palindromic.

\begin{itemize}
	\item Number of digits in integer $x$:
$$n = \lfloor \log_10(x) \rfloor + 1$$

	\item Least significant (last) digit in integer $x$:
$$x \mod 10$$

	\item Most significant digit in integer $x$:
$$\frac{x}{10^{(\lfloor \log_10 x \rfloor + 1)}}$$

	\item Remove most significant digit:
$$x = x \mod 10^{\text{num\_digits - 1}}$$

	\item Remove least significant digit:
$$x = x / 10$$
\end{itemize}

\paragraph{Generate uniform random numbers in range $[a,b]$ given random 0/1 values}
Same as in range $[0, b-a]$.

Easy if $[b-a] = 2^i - 1$ for some $i$: Generate each bit of a $i$-bit number.

Otherwise, search smallest $i$ such that $2^i - 1 \geq b - a$, and redo generation if out of range.


\paragraph{Rectangle intersection (parallel to x- and y-axis)}
Focus on when rectangles \textbf{don't} intersect, handle x- and y-dimension separately.

\subsection{Arrays}
\subsection{Linked Lists}
\subsection{Stacks}
\subsection{Queues}
\subsection{Binary Trees}
\subsection{Heaps}
\subsection{Searching}
\subsection{Hash Tables}
\subsection{Sorting}
\subsection{Binary Search Trees}
\subsection{Recursion}
\subsection{Dynamic Programming}
\subsection{Greedy Algorithms}
\subsection{Invariants}
\subsection{Graphs}

\subsection(Computational Geometry}
\paragraph{Compute area within polygon given its vertices}
Shoelace formula

\section{Cool C++ features}

\section{Cool Python Features}

% ---- Bibliography ----
%
% BibTeX users should specify bibliography style 'splncs04'.
% References will then be sorted and formatted in the correct style.
%
 \bibliographystyle{splncs04}
 \bibliography{document}


\end{document}
